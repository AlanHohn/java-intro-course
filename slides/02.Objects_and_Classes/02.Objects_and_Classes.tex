\documentclass{beamer} 
\usepackage{dcolumn}
\usepackage{listings}
\lstset{basicstyle=\scriptsize, showspaces=false, showtabs=false, showstringspaces=false, breaklines, prebreak=..., tabsize=2}
\usepackage{graphicx}
\usepackage{textcomp}
\usepackage{url}
\usepackage{hanging}
\urlstyle{rm}
\newcolumntype{.}{D{.}{.}{-1}}
\newcolumntype{d}[1]{D{.}{.}{#1}}
\usetheme{JuanLesPins}
\setbeamertemplate{footnote}{%
  \hangpara{2em}{1}%
  \makebox[2em][l]{\insertfootnotemark}\footnotesize\insertfootnotetext\par%
}
\beamertemplatenavigationsymbolsempty

\title[]{Objects and Classes}
\subtitle{Introduction to Java}
\author{Alan Hohn\\
\texttt{Alan.M.Hohn@lmco.com}}
\date{25 July 2013}

\begin{document}

\begin{frame}
   \titlepage
\end{frame}

\begin{frame}
   \frametitle{Contents}
   \tableofcontents[]
\end{frame}

\section{Brief Review}
\begin{frame}
\frametitle{Course Contents}
\begin{itemize}
\item Getting started writing Java programs (last time)
\item Java programming language basics (today is 1 of 4 sessions)
\item Packaging Java programs (1 session)
\item Core library features (6 sessions)
\item Java user interfaces (2 sessions)
\end{itemize}
\end{frame}

\begin{frame}
\frametitle{Java Portability}
\begin{itemize}
\item A key motivation for Java is portability (ability to run on different platforms without recompiling)
\item This is accomplished by compiling for a ``virtual machine'' with its own instruction set
\begin{itemize}
\item Known, appropriately, as the Java Virtual Machine (JVM)
\item JVM instructions are called ``bytecode''
\item Looks like assembly / machine language, but with added features like virtual function calls
\end{itemize}
\item The JVM provides a way to run bytecode on a specific operating system / machine instruction set
\item Generally, the same bytecode can be run unmodified on any JVM
\end{itemize}
\end{frame}

\begin{frame}[fragile]
\frametitle{Our Basic Java Example}
\lstinputlisting[language=Java]{../examples/src/org/anvard/introtojava/HelloName.java}
\end{frame}

\begin{frame}
\frametitle{Java Memory Management}
\begin{itemize}
\item That program raises questions about memory management
\begin{itemize}
\item We used the \texttt{new} keyword to make some objects
\item We didn't ``free'' those objects, set them to \texttt{null}, or otherwise worry about them
\end{itemize}
\item In Java, new objects are allocated on the ``heap''
\begin{itemize}
\item The JVM manages the heap
\item When the heap (or part of it) gets full, the JVM does ``garbage collection''
\item This means identifying objects that are no longer referenced from live code and freeing the memory
\end{itemize}
\item Object allocation happens all the time in Java
\begin{itemize}
\item Most objects are short-lived
\item For example, the \texttt{readLine} method and the \texttt{+} operator both instantiated new string objects
\item Modern JVMs are optimized for lots of short-lived objects, so this is surprisingly performant
\end{itemize}
\end{itemize}
\end{frame}

\begin{frame}
\frametitle{This Time}
\begin{itemize}
\item Java Objects and Classes
\item Brief overview of Object-Oriented Programming
\item Distinction between primitive types and objects
\item Distinction between static and instance variables
\end{itemize}
\end{frame}

\section{Object-Oriented Programming}
\begin{frame}
\frametitle{OOP for Programmers}
\begin{itemize}
\item Every ``new'' idea in Object-Oriented Programming (OOP) is not really new
\item The basic design purpose behind OOP is encapsulation
\begin{itemize}
\item Information should be held in the smallest possible scope
\item Simplifies maintenance, analysis of behavior, and reuse
\end{itemize}
\item OOP takes the \texttt{record} or \texttt{struct} and adds behavior to it
\begin{itemize}
\item Keep the code that operates on a piece of data together with the data
\item Control the scope of both code and data so other code can only access through a defined interface
\end{itemize}
\item OOP lets us play other tricks that we will see later, but the basic idea is still improved encapsulation, because the data and the code that operates on it ``live'' inside the same construct
\end{itemize}
\end{frame}

\begin{frame}
\frametitle{Objects and Classes in Java}
\begin{itemize}
\item (Almost) everything in Java is an object
\item All the code we write will be in a ``class''
\begin{itemize}
\item A class defines what an object ``looks like'': its unique name, its behavior, what data it stores
\item There can be many objects that have the same class; each object is known as an ``instance'' of the class
\item All share the same behavior, but have (mostly) independent data
\end{itemize}
\item Classes have ``fields'' that hold data, and ``methods'' that specify behavior
\end{itemize}
\end{frame}

\begin{frame}[fragile]
\frametitle{Simple Class}
\lstinputlisting[language=Java, lastline=19]{../examples/src/org/anvard/introtojava/designpatterns/Person.java}
\end{frame}

\begin{frame}[fragile]
\frametitle{Simple Class (continued)}
\lstinputlisting[language=Java, firstline=21, lastline=36]{../examples/src/org/anvard/introtojava/designpatterns/Person.java}
\lstset{language=Java}
\begin{lstlisting}
  ...
    
}
\end{lstlisting}
\end{frame}

\section{OOP in Java}
\begin{frame}
\frametitle{Where Does a Class Reside?}
\begin{itemize}
\item When a class is \emph{loaded}, its \texttt{.class} file is read into memory in the permanent generation (PermGen)
\item When an object is \emph{instantiated} (e.g. \texttt{Person p = new Person()}), the JVM allocates space on the heap for its \emph{instance data} (fields)
\item The class stays in memory for the life of the JVM
\item The instance stays on the heap until it is garbage collected
\end{itemize}
\end{frame}

\begin{frame}
\frametitle{Object References}
\begin{itemize}
\item An object can refer to another object, but it \emph{does not} keep its own copy of that object's data
\begin{itemize}
\item All objects are instantiated on the heap
\item A field just contains a \emph{reference} to the other object
\item The \texttt{this} keyword is a reference to the ``current'' instance
\end{itemize}
\item Object references are used for parameters, too
\begin{itemize}
\item Java is pass-by-value, but what is passed in the case of an object parameter is always a reference to the object (similar to a pointer)
\item Only the object reference lives on the stack; the actual object is on the heap
\item Objects modified inside methods retain their modifications when the method returns\footnote{Except for primitive wrapper classes, see below}
\end{itemize}
\end{itemize}
\end{frame}

\section{Java Basic Data Types}
\begin{frame}[fragile]
\frametitle{Primitives}
\lstset{language=Java}
\begin{lstlisting}
  private Integer id;
\end{lstlisting}
\begin{itemize}
\item This is an object reference that is an instance of \texttt{java.lang.Integer}, a class that is part of the JVM
\item This means it is allocated on the heap and must eventually be garbage collected
\item It also means it is allowed to be \texttt{null}, which can be useful in some cases
\end{itemize}
\lstset{language=Java}
\begin{lstlisting}
  private int id2;
\end{lstlisting}
\begin{itemize}
\item This is a simple integer called a primitive
\item It is more efficient in storage, but it cannot be null and does not have fields or methods
\item The types of primitives are \texttt{byte}, \texttt{short}, \texttt{int}, \texttt{long}, \texttt{float}, \texttt{double}, \texttt{boolean}, and \texttt{char}
\item All primitives in Java are signed
\end{itemize}
\end{frame}

\begin{frame}[fragile]
\frametitle{Primitives Example}
\lstinputlisting[language=Java, lastline=17]{../examples/src/org/anvard/introtojava/PrimitivesExample.java}
\end{frame}

\begin{frame}[fragile]
\frametitle{Primitives Example}
\lstinputlisting[language=Java, firstline=18]{../examples/src/org/anvard/introtojava/PrimitivesExample.java}
\end{frame}

\begin{frame}[fragile]
\frametitle{Primitives Example}
\lstset{language=}
\begin{lstlisting}
i: 0
objectI: null
objectI: 5
objectI: 10
i: 20
i: 30
val: 30       <-- Note no change in 'val'
\end{lstlisting}
\end{frame}

\begin{frame}[fragile]
\frametitle{Strings}
\begin{itemize}
\item Strings are instances of \texttt{java.lang.String}
\item The actual characters of the string typically exist on the heap
\item Strings are immutable
\item Modifying a string typically results in memory allocation for a new string
\begin{itemize}
\item This has important implications for performance, discussed later
\item One exception is \emph{string literals}
\item These are allocated in a single pool and reused
\end{itemize}
\end{itemize}
\lstset{language=Java}
\begin{lstlisting}
String a = "abc"; // a points to a string in the pool
String b = a; // Reuse: b points to the same string in the pool
String c = a + b; // Brand new string "abcabc" allocated on the heap
\end{lstlisting}
\end{frame}

\begin{frame}[fragile]
\frametitle{Primitive Wrapper Classes}
\begin{itemize}
\item Each primitive has a corresponding ``wrapper'' class (e.g. int and java.lang.Integer)
\item These are objects: they exist on the heap, they can be null, they have methods
\item However, they are also immutable
\begin{itemize}
\item Changing the `value' results in changing the instance to point to a new value
\item Might be a reused value from a pool or a new one
\end{itemize}
\item Java SE 5 and later automatically converts between primitives and the corresponding wrapper class (a.k.a. autoboxing)
\end{itemize}
\lstset{language=Java}
\begin{lstlisting}
Integer a = new Integer(5); // Always a new instance
a = 6; // Auto-boxing, illegal before Java SE 5
a = Integer.valueOf(6) // This is what really happens
\end{lstlisting}
\end{frame}

\begin{frame}
\frametitle{Static and Instance Variables}
\begin{itemize}
\item The \texttt{static} keyword marks a field or method as belonging to the class as a whole, not one particular instance
\item Only one copy of a static field exists for the whole class
\begin{itemize}
\item The static field exists in PermGen, but it might be an object reference to an object on the heap
\item While the instance points to a live object, that object is never garbage collected (because the class is in PermGen and never goes away)
\end{itemize}
\item Static fields and methods can be accessed \emph{and should be accessed} without referring to an object instance
\item Static fields and methods must be handled carefully in multi-threaded environments
\end{itemize}
\end{frame}

\begin{frame}[fragile]
\frametitle{Static Example}
\lstinputlisting[language=Java]{../examples/src/org/anvard/introtojava/StaticExample.java}
\end{frame}

\begin{frame}
\frametitle{Next Time}
\begin{itemize}
\item Java Control Flow
\item Operators
\item Exception Handling
\end{itemize}
\end{frame}


\begin{frame}
\frametitle{Credit in LMPeople}
\begin{center}
Last Time: LMPeople Course Code: 071409ILT01

This Time: LMPeople Course Code: 071409ILT03
\end{center}
\end{frame}

\end{document}

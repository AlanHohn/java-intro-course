\documentclass{beamer} 
\usepackage{dcolumn}
\usepackage{listings}
\lstset{basicstyle=\scriptsize, showspaces=false, showtabs=false, showstringspaces=false, breaklines, prebreak=..., tabsize=2}
\usepackage{graphicx}
\usepackage{textcomp}
\usepackage{url}
\usepackage{hanging}
\urlstyle{rm}
\newcolumntype{.}{D{.}{.}{-1}}
\newcolumntype{d}[1]{D{.}{.}{#1}}
\usetheme{JuanLesPins}
\setbeamertemplate{footnote}{%
  \hangpara{2em}{1}%
  \makebox[2em][l]{\insertfootnotemark}\footnotesize\insertfootnotetext\par%
}
\beamertemplatenavigationsymbolsempty

\title[]{Language Basics}
\subtitle{Introduction to Java}
\author{Alan Hohn\\
\texttt{Alan.M.Hohn@lmco.com}}
\date{29 August 2013}

\begin{document}

\begin{frame}
   \titlepage
\end{frame}

\begin{frame}
   \frametitle{Contents}
   \tableofcontents[]
\end{frame}

\section{Brief Review}
\begin{frame}
\frametitle{Course Contents}
\begin{itemize}
\item Getting started writing Java programs (complete)
\item Java programming language basics (today is \#2 of 4 sessions)
\item Packaging Java programs (1 session)
\item Core library features (6 sessions)
\item Java user interfaces (2 sessions)
\end{itemize}
\end{frame}

\begin{frame}[fragile]
\frametitle{Our Basic Java Example}
\lstinputlisting[language=Java]{../examples/src/org/anvard/introtojava/HelloName.java}
\end{frame}

\begin{frame}
\frametitle{This Time}
\begin{itemize}
\item Method Calls and Field References
\item Control Flow
\item Operators
\item Object Equality
\end{itemize}
\end{frame}

\section{Method Calls and Field References}
\begin{frame}
\frametitle{The \texttt{this} keyword}
\begin{itemize}
\item Java methods and fields exist inside a class
\begin{itemize}
\item Unless static, methods and fields belong to a particular instance of the class (to an object)
\item Java uses \texttt{this} to refer back to the current object instance
\end{itemize}
\item In Java, all references are ``scoped''
\end{itemize}
\end{frame}


\begin{frame}[fragile]
\frametitle{Scoped references}
\begin{itemize}
\item Dot notation is used in Java to ``enter'' a level of scope
\item The compiler searches outward for an unqualified reference
\item The scope ends with the class (there are no global fields or methods)
\item In other words, it is OK to use the keyword \texttt{this}, but not necessary unless avoiding ambiguity
\lstset{language=Java}
\begin{lstlisting}
public void setValue(int value) {
  this.value = value;
}
\end{lstlisting}
\item Field references and method calls can be ``chained''
\lstset{language=Java}
\begin{lstlisting}
// out is a (static) field in the System class
// println is a method in the PrintStream class
System.out.println("Hello, world!");
\end{lstlisting}
\end{itemize}
\end{frame}

\begin{frame}[fragile]
\frametitle{Method Definitions}
\begin{itemize}
\item Java methods have four parts
\begin{itemize}
\item Return type (may be void)
\item Identifier (name)
\item Parameter list
\item Exceptions
\end{itemize}
\item No two methods in the same class may have the same name and parameter list
\item Methods may also have qualifiers such as \texttt{public} or \texttt{static}
\end{itemize}
\lstset{language=Java}
\begin{lstlisting}
public int add(Integer left, Integer right) 
  throws OverflowException;
\end{lstlisting}
\end{frame}

\begin{frame}[fragile]
\frametitle{Method Calls}
\begin{itemize}
\item The instance on which the method operates is like an implicit first parameter
\begin{itemize}
\item It is available to the method using \texttt{this}
\item It is used implicitly when the method refers to instance variables
\end{itemize}
\item Objects returned from methods may be ``chained''
\item Method calls must use parentheses even if there are no parameters, to avoid ambiguity with field references
\item Instantiating an object with \texttt{new} is like calling a constructor method; it returns an object of the instantiated type
\end{itemize}
\lstset{language=Java}
\begin{lstlisting}
System.out.println("Hello, world!");
String name = new BufferedReader(
  new InputStreamReader(System.in)).readLine();
\end{lstlisting}
\end{frame}

\begin{frame}[fragile]
\frametitle{Quick Note on Naming}
\begin{itemize}
\item The compiler searches for an unqualified reference in the current scope
\item The \texttt{import} statement brings a class into the current scope so it can be used unqualified (without its full package name)
\item By convention, Java classes are UpperCamelCase, while packages, fields, methods and variables are lowerCamelCase
\item This is done to avoid hiding names unnecessarily
\end{itemize}
\lstset{language=Java}
\begin{lstlisting}
// import System; is implied in any Java program
public String system() { return "Laptop"; }
public String system = "Desktop";
System.out.println("Hello, " + system()); // Hello, Laptop
System.out.println("Hello, " + system); // Hello, Desktop
System.out.println("Hello, " + System); // Compiler error
\end{lstlisting}
\end{frame}

\section{Control Flow}
\begin{frame}
\frametitle{Java Control Flow}
\begin{itemize}
\item Java has the expected set of flow control keywords
\begin{itemize}
\item \texttt{if} - \texttt{ else if } - \texttt{else}
\item \texttt{switch} - \texttt{case} - \texttt{default}
\item \texttt{while} and \texttt{do} - \texttt{while}
\item \texttt{for} (including an enhanced version for iterators)
\end{itemize}
\item \texttt{if} conditionals and \texttt{for} expressions must be in parentheses
\item Curly braces ( \{ \} ) are used to group multiple statements; they are optional for a single-statement block but are recommended
\item \texttt{break} works with \texttt{switch}, \texttt{for}, \texttt{while}, or \texttt{do-while}
\item \texttt{continue} works with \texttt{for}, \texttt{while}, or \texttt{do-while}
\item Labels are supported for \texttt{break} and \texttt{continue}
\item \texttt{return} can appear anywhere in a method
\item There is no \texttt{goto}
\end{itemize}
\end{frame}

\begin{frame}[fragile]
\frametitle{Control Flow Example}
\lstset{language=Java}
\begin{lstlisting}
class BreakWithLabelDemo {
  public boolean search2dArray(int[][] arrayOfInts, 
    int searchfor) {
    boolean foundIt = false;
    search:
    for (int i = 0; i < arrayOfInts.length; i++) {
      for (int j = 0; j < arrayOfInts[i].length; j++) {
        if (arrayOfInts[i][j] == searchfor) {
          foundIt = true;
          break search; // Or just 'return true;'
        }
      }
    }
    return foundIt;
  }
}
\end{lstlisting}
\end{frame}

\begin{frame}[fragile]
\frametitle{Enhanced \texttt{for}}
\begin{itemize}
\item Some Java classes implement an interface called \texttt{Iterable}
\item This includes arrays and most of the built-in collection classes
\item This means they will provide an ``iterator'' for looping through multiple objects
\item Java provides a cleaner version of \texttt{for} for \texttt{Iterable} classes
\item This version can also be more performant for some collections
\end{itemize}
\lstset{language=Java}
\begin{lstlisting}
class EnhancedForDemo {
  public static void main(String[] args){
     int[] numbers = 
       {1,2,3,4,5,6,7,8,9,10};
     for (int item : numbers) {
       System.out.println("Item is: " + item);
     }
  }
}
\end{lstlisting}
\end{frame}

\section{Operators and object equality}
\begin{frame}
\frametitle{Java Operators}
\begin{itemize}
\item Java has the expected set of operators (\texttt{+ - * / \%})
\item Similar to C, there is a difference between bitwise operators (\texttt{\& | \^}) and comparison operators (\texttt{\&\& ||})
\item Unlike C, Java has no ``unsigned'' values, but in addition to the shift operators (\texttt{<< >>}) there is an ``unsigned'' shift right \texttt{>>>}.
\item Java has clever compound operators like C (e.g. \texttt{++ -- +=}) and the ternary operator \texttt{(? :)}
\item The comparison operators (\texttt{< <= > >=}) only work for primitive types and their wrapper classes
\item There is no exponentiation operator; use \texttt{Math.pow()}
\end{itemize}
\end{frame}

\begin{frame}[fragile]
\frametitle{Widening and Narrowing Conversions}
\begin{itemize}
\item Java is a strongly typed language
\item However, operators can mix types under certain circumstances
\item Java will automatically perform ``widening'' conversions ($\texttt{byte} \rightarrow \texttt{short} \rightarrow \texttt{int} \rightarrow \texttt{long} \rightarrow \texttt{float} \rightarrow \texttt{double}$)
\item Narrowing conversion can lose information, so Java requires an explicit cast
\lstset{language=Java}
\begin{lstlisting}
double d = 123.45; float f = (float) d;
\end{lstlisting}
\item When mixing operators, the resulting value will be the `widest' required
\item For example, for the division operator \texttt{/}, the result can be an integer (for integer / integer) or a floating-point value (for floating-point or mixed types)
\end{itemize}
\end{frame}

\begin{frame}
\frametitle{Equality Operators}
\begin{itemize}
\item Java has two kinds of equality: reference equality (\texttt{o1 == o2}) and object equality (\texttt{o1.equals(o2)})
\item Reference equality means the two references refer to exactly the same object in memory
\item Object equality means the objects are semantically the same (mean the same thing)
\item For classes you create, you can (and often should) define your own rules for object equality by making your own \texttt{equals()} method
\item Later we will talk about \texttt{hashCode()}, an important method that's often seen together with \texttt{equals()}
\end{itemize}
\end{frame}

\begin{frame}[fragile]
\frametitle{Equality example}
\lstset{language=Java}
\begin{lstlisting}
public class Person {
  public String firstName;
  public String lastName;
  public boolean equals(Object obj) {
    if (null == obj) return false;
    if (obj.getClass() != this.getClass()) return false;
    Person other = (Person)obj;
    return other.firstName == this.firstName && 
        other.lastName == this.lastName;
  }
}

Person p1 = new Person(); 
p1.firstName = "Joe";
p1.lastName = "Smith";
Person p2 = new Person(); 
p2.firstName = "Joe";
p2.lastName = "Smith";

p1 == p2; // false
p1.equals(p2) // true
\end{lstlisting}

\end{frame}



\section{Next Time}
\begin{frame}
\frametitle{Next Time}
\begin{itemize}
\item Exception Handling
\item Checked and Unchecked Exceptions
\item Try-With-Resources
\end{itemize}
\end{frame}


\begin{frame}
\frametitle{Credit in LMPeople}
\begin{center}
LMPeople Course Code: 071409ILT04
\end{center}
\end{frame}

\end{document}
